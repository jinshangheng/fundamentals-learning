\documentclass[../main.tex]{subfiles}
\begin{document}
    \subsection{第2章}
    \subsubsection{操作系统提供的服务}
    操作系统提供一组服务,通常包括:用户界面、程序执行、I/O操作、文件系统操作、通信、错误检测、资源分配、记账、保护与安全。
    \subsubsection{系统调用}
    一些通常由C/C++编写的操作系统服务接口。但是程序员更倾向于使用API而不是系统调用接口,原因是前者具有更强的移植性,适用于一类系统和计算机。
    系统调用的类型一般包括进程控制、文件管理、设备管理、信息维护、通信、保护。标准库可以被视为连接用户模式和内核模式的桥梁。
    \subsubsection{操作系统的结构}
    松散自由的结构会导致系统容易发生错误或受到恶意程序的侵害,但是可以节省一部分的开销,这也是早期主流的系统设计。
    再之后,分层方法、混合系统成为了主导。它们通过层层约束和接口传递,保证了系统的安全,避免了各层的定义与交互的问题。
    \subsubsection{习题}
    \noindent 2.1 操作系统提供的服务和功能可以分为两大类,分别是什么? \\
    A: 用户功能和系统功能。前者提供解决具体问题的接口和方法,后者维持系统高效、稳定运行。 \\
    2.4 操作系统文件管理的五个主要功能是什么? \\
    A: 增、删、读、写、重定位。 \\
    2.8 为什么机制和策略的分离是可取的。 \\
    A: 机制决定怎么做,策略决定做什么。策略可随时间地点而改变,而底层机制对于策略改变的不敏感可以更大程度提高系统的稳定性。\\
    2.12 iOS和Android有何区别? \\
    A: Android的内核是Linux内核,而iOS不是。iOS在设计的时候重视交互设计,将独有的触控交互作为最上层,媒体服务次之,而Android更贴近于传统的操作系统,以架构和应用程序为上层。
\end{document}