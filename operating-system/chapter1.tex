\documentclass[../main.tex]{subfiles}
\begin{document}
    \subsection{第1章}
    \subsubsection{定义操作系统}
    我们可以把操作系统看作\emph{资源分配器}。
    一个操作系统的结构包括内核到中间件再到应用程序层次。其中中间件是为应用程序开发人员提供的软件框架。
    \subsubsection{冯诺伊曼体系结构}
    运算器、控制器、存储器、输入设备、输出设备。
    一个典型的指令执行周期:首先从内存中获取指令,并存到指令寄存器,最终该指令被解码执行。
    \subsubsection{I/O}
    在IO开始的时候,设备驱动程序家在设备控制器的适当寄存器。相应地,设备控制器检查这些寄存器内容,以便决定采取什么操作。
    一旦完成数据传输,设备控制器就会通过中断通知设备驱动程序。这种IO模式对于大数据的移动,例如磁盘IO会带来更大的系统开销。
    所以为了解决这个问题,可以采用直接内存访问(DMA),只需要设置好缓冲、指针、计数器之后,无需CPU的干预,可以直接完成大量数据的IO,中间无需CPU的干预。
    \subsubsection{习题}
    \noindent1.1 多个用户同时共享同一个系统,会导致什么安全问题?\\
    A: 窃取或复制或篡改其他用户的数据;没有合理分配的预算来使用资源。 \\
    1.3 在何种环境下,分时系统优于PC或但用户工作站? \\
    A: 硬件速度足够快,任务相对巨大。 \\
    1.5 集群系统与多处理器系统有何不同? \\
    A: 集群系统是通过网络系统耦合多个计算机来完成同一个计算任务的系统。多处理器系统是同一物理实体包含多个处理器。
    前者的耦合度低于后者,后者的经济效益高于前者。 \\
    1.11 许多SMP系统有不同层次的缓存,有的缓存是为单个处理核专用的,而有的缓存是为所有处理核共用的。为什么这样设计缓存? \\
    A: 各个处理核的私有缓存是为了保证单核运算效率,共用缓存是为了保证各个处理核可以协同使用同一个内存和数据结构。
\end{document}